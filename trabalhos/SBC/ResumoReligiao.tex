\documentclass[12pt]{article}

\usepackage{sbc-template}

\usepackage{graphicx,url}

%\usepackage{indentfirst}

\usepackage[brazil]{babel}   
%\usepackage[latin1]{inputenc}  
\usepackage[utf8]{inputenc}  
% UTF-8 encoding is recommended by ShareLaTex


\sloppy

\title{Capítulo 1: O Problema}

\author{Túlio N. P. Lopes\inst{1},Guilherme Assis Oliveira\inst{1}}


\address{Instituto de Ciências Exatas e Informática -- Pontifica Universidade Católica de Minas Gerais\\
	Belo Horizonte -- MG -- Brasil
\email{tulionp.lopes@gmail.com, guilhermeassisoliveira@gmail.com}
}

\begin{document} 
	
	\maketitle
	\section{Resumo:}
	
	
	Neste capítulo, que serve de introdução para o restante do livro, o autor Erich Fromm, faz uma analise psico-social, avaliando as formas de agir do ser humano do séc. XX comparadas às formas que agiam os dos séculos passados. Ele acredita que o homem moderno se torna cada vez mais adepto aos ideias iluministas, que pregavam a "razão" como ferramenta mais importante para a evolução da sociedade.
	
	Apesar de parecer algo positivo para o futuro da sociedade humana, isso tem nos tornado de certa forma inquietos. Afinal,  mesmo tendo o controle sobre a natureza aumentado, o ser humano tem perdido sua essência. Ele não mais se preocupa com o que faz dele homem, que é a capacidade de viver como tal, mas sim se prende às questões externas ao seu ser. O que faz dele um escravo de sua própria criação, a tecnologia. Assim se faz uma crise das ideias iluministas, onde o homem não acredita mais em progresso social, ele se vê mais "realista", sem a esperança de que caminhamos rumo a um futuro brilhante a todos.
	
	Fromm acredita que há uma saída desse problema. Ele prega que para que isso seja solucionado, o homem não deve mais se basear apenas na razão humana para formular normas morais e éticas. O que deve ser feito é estudar o passado humano, a própria "humanidade", ou seja, aquilo que nos torna o que somos. Dessa forma poderíamos construir uma base social tão válidas quanto as que temos baseadas na razão, afinal é o instinto animal da raça que nos faz assim, ele é que nos trouxe ao presente da forma que chegamos.
	
	Não podemos conviver se não olharmos para dentro dos nossos seres. Não podemos deixar que sejamos regidos por regulamentos, sejam eles éticos ou morais, que foram criados de outra forma senão analisando nosso interior como homens. A sociedade precisa deixar o relativismo e passar a analisar a natureza humana a fim de definir o que deveríamos ou não fazer e a forma que agimos. Devemos basear nossa conduta no nosso próprio interior, e não naquilo que nos cerca. 
	
	%\begin{abstract}
	%	This meta-paper describes the style to be used in articles and short papers
	%	for SBC conferences. For papers in English, you should add just an abstract
	%	while for the papers in Portuguese, we also ask for an abstract in
	%	Portuguese (``resumo''). In both cases, abstracts should not have more than
	%	10 lines and must be in the first page of the paper.
	%\end{abstract}
	
	%begin{resumo} 
	%	Este meta-artigo descreve o estilo a ser usado na confecção de artigos e
	%	resumos de artigos para publicação nos anais das conferências organizadas
	%	pela SBC. É solicitada a escrita de resumo e abstract apenas para os artigos
	%	escritos em português. Artigos em inglês deverão apresentar apenas abstract.
	%	Nos dois casos, o autor deve tomar cuidado para que o resumo (e o abstract)
	%	não ultrapassem 10 linhas cada, sendo que ambos devem estar na primeira
	%	página do artigo.
	%\end{resumo}
	
	
	
	
	
	
	
	

	
	%\section{References}
	
	%Bibliographic references must be unambiguous and uniform.  We recommend giving
	%the author names references in brackets, e.g. \cite{knuth:84},
	%\cite{boulic:91}, and \cite{smith:99}.
	
	%The references must be listed using 12 point font size, with 6 points of space
	%before each reference. The first line of each reference should not be
	%indented, while the subsequent should be indented by 0.5 cm.
	
	%\bibliographystyle{sbc}
	%\bibliography{sbc-template}
	
\end{document}
