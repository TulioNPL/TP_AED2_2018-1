\documentclass[12pt]{article}

\usepackage{sbc-template}

\usepackage{graphicx,url}

%\usepackage{indentfirst}

\usepackage[brazil]{babel}   
%\usepackage[latin1]{inputenc}  
\usepackage[utf8]{inputenc}  
% UTF-8 encoding is recommended by ShareLaTex


\sloppy

\title{Somatório}

\author{Túlio N. P. Lopes\inst{1}}


\address{Instituto de Ciências Exatas e Informática -- Pontifica Universidade Católica de Minas Gerais\\
	Belo Horizonte -- MG -- Brasil
\email{tulionp.lopes@gmail.com}
}

\begin{document} 
	
	\maketitle
	
	%\begin{abstract}
	%	This meta-paper describes the style to be used in articles and short papers
	%	for SBC conferences. For papers in English, you should add just an abstract
	%	while for the papers in Portuguese, we also ask for an abstract in
	%	Portuguese (``resumo''). In both cases, abstracts should not have more than
	%	10 lines and must be in the first page of the paper.
	%\end{abstract}
	
	%begin{resumo} 
	%	Este meta-artigo descreve o estilo a ser usado na confecção de artigos e
	%	resumos de artigos para publicação nos anais das conferências organizadas
	%	pela SBC. É solicitada a escrita de resumo e abstract apenas para os artigos
	%	escritos em português. Artigos em inglês deverão apresentar apenas abstract.
	%	Nos dois casos, o autor deve tomar cuidado para que o resumo (e o abstract)
	%	não ultrapassem 10 linhas cada, sendo que ambos devem estar na primeira
	%	página do artigo.
	%\end{resumo}
	
	
	Somatório é a soma arbitrária de termos desenvolvida e estudada na área da matemática. Através do somatório pode se criar formas de representar a soma regrada e sistemática de termos, resultando em uma infinidade de formas se criar um um somatório. A letra grega sigma é o símbolo que representa esse conceito, sendo usando sempre acompanhado por um índice inicial, abaixo de sigma e um índice final, acima de sigma, por exemplo:
	
	$\sum_{i=1}^{n}{x_i} = x_1+x_2+x_3+ \cdots +x_n $
	
	Nessa fórmula pode-se notar o índice inicial, no caso $i = 1$, e o índice final, $n$, respectivamente abaixo e acima de sigma($\sum$). Ao lado é possível descrever as regras do somatório, que no caso representado acima soma termo a termo incrementando 1 a cada soma, ou seja, o somatório inicia de 1 e termina ao chegar a $n$.
	
	Uma das fórmulas de somatório mais conhecidas é a soma de Gauss, desenvolvida pelo próprio entre os séculos XVIII e XIX. Acredita-se que durante uma aula de matemática um professor pediu aos alunos que somassem todos os números de um a cem, surpreendentemente em poucos minutos Gauss atingiu a resposta correta. Ele observou que se fossem somados o primeiro e o último numero, o resultado seria 101, o mesmo para o segundo e antepenúltimo e assim sucessivamente. Tendo isso em mente Gauss simplesmente multiplicou esse resultado pela metade dos termos, no caso 50, atingindo o resultado correto. Esse cálculo ficou conhecido como soma de Gauss, e é representado pelas seguintes fórmulas:
	
	$\sum_{i=1}^{100}{i} = 1+2+3+\dots+100$  
	
	
	ou de forma mais simples,
	
	
	$\frac{n(n+1)}{2}$
	
	Isso demonstra que o somatório nada mais é do que uma forma de representar uma soma matematicamente, sem que haja a necessidade de descrever em palavras como deve ser feita. Apesar de parecer uma solução, ele é usado como um tipo de linguagem matemática para demonstrar como uma soma deve ser feita, indicando o ponto inicial, o ponto final e as regras da soma em si.
	
	
	
	
	

	
	%\section{References}
	
	%Bibliographic references must be unambiguous and uniform.  We recommend giving
	%the author names references in brackets, e.g. \cite{knuth:84},
	%\cite{boulic:91}, and \cite{smith:99}.
	
	%The references must be listed using 12 point font size, with 6 points of space
	%before each reference. The first line of each reference should not be
	%indented, while the subsequent should be indented by 0.5 cm.
	
	%\bibliographystyle{sbc}
	%\bibliography{sbc-template}
	
\end{document}
